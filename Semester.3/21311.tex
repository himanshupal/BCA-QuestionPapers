\documentclass[12pt]{article}
\usepackage{amsmath}
\usepackage{fullpage}
\begin{document}
\pagestyle{empty}
\hyphenpenalty=10000

\begin{center}

	\textsf{
		\textbf{
			\LARGE Paper Code : 21311\\
			\normalsize F-411\\
			B.C.A. (IIIrd Semester)\\
			Examination, 2019-20\\
			(New Course)\\
			\small{COMPUTER ORIENTED NUMERICAL ANALYSIS\\}
		}
		Paper-BCA-301-N
	}

	\begin{minipage}{0.49\linewidth}
		\raggedright\footnotesize Time : 3 Hours ]
	\end{minipage}
	\begin{minipage}{0.49\linewidth}
		\raggedleft\footnotesize [ Maximum Marks : 70
	\end{minipage}

\end{center}

\rule{0.9\linewidth}{0.1mm}

\begin{description}
	\item[\textit{Note }:-]
	\begin{enumerate}
		\item Attempt any \textit{five} questions. All questions carry equal marks.
		\item Calculator is Allowed.
	\end{enumerate} 
\end{description}

\begin{enumerate}

	\item[1.]
	\begin{enumerate}
		\item If 0.333 is the approximate value of $\frac{1}{3}$, find absolute, relative and percentage error.
		\item Find the product of numbers 56.64 and 12.4 which are both correct to the significant digits given.
	\end{enumerate}

	\item[2.]
	\begin{enumerate}
		\item What are the various floating point representation of members ? Explain.
		\item Explain bisection method for root finding also. Find real root of $e^x = 3x$ by bisection method.
	\end{enumerate}

	\item[3.]
	\begin{enumerate}
		\item Solve by iteration method :
			\begin{equation*}
				\sin x = \frac{(x+1)}{(x-1)}
			\end{equation*}
		\item Given that :
			\begin{center}
				\begin{tabular}{|c|c|c|c|c|c|c|}
					\hline
					x      & 1 & 2 & 3 & 4  & 5  & 6   \\
					\hline
					$y(x)$ & 0 & 1 & 8 & 27 & 64 & 125 \\
					\hline
				\end{tabular}
			\end{center}
			Find the value of $y(2.5).$
		\item What do you mean by Inter Polation ? Explain.
	\end{enumerate}

	\item[4.]
	\begin{enumerate}
		\item Write various merits and demerits of Lagrange's formula.
		\item Evaluate :
			\begin{equation*}
				\int_0^6\frac{dx}{1+x^2}\ \textnormal{by using}
			\end{equation*}
			\begin{enumerate}
				\item Simpson's one-third rule
				\item Simpson's three-eight rule
				\item Trapezoidal's rule
				\item Weddle's rule
			\end{enumerate}
	\end{enumerate}

	\item[5.]
	\begin{enumerate}
		\item What do you mean by Single-step Method and multi-Step Method for numerical solution of differential equations ?
		\item Explain Picard's method of successive approximation.
	\end{enumerate}

	\item[6.]
	\begin{enumerate}
		\item Explain Runge-Kutta Method with a suitable example.
		\item What is Automative Error Monitoring ?
		\item Explain Taylor's method for I order differential equation.
	\end{enumerate}

	\item[7.]
	\begin{enumerate}
		\item Fit a straight line to the folowing data regarding $x$ as the independent variable :
			\begin{center}
				\begin{tabular}{|c|c|c|c|c|c|c|}
					\hline
					$x$ & 1    & 2   & 3   & 4   & 5   & 6  \\
					\hline
					$y$ & 1200 & 900 & 600 & 200 & 110 & 50 \\
					\hline
				\end{tabular}
			\end{center}
		\item Fit an equation of the form $y=ae^{bx}$ to the following data by the method of least squares :
			\begin{center}
				\begin{tabular}{|c|c|c|c|c|}
					\hline
					$x$ & 1    & 2   & 3   & 4    \\
					\hline
					$y$ & 1.65 & 2.7 & 4.5 & 7.35 \\
					\hline
				\end{tabular}
			\end{center}
	\end{enumerate}

	\item[8.] Write any \textit{two} short notes on the following:
	\begin{enumerate}
		\item Regression Analysis
		\item Statistical Quality Control
		\item Computer Arithmetic
	\end{enumerate}

\end{enumerate}

\end{document}